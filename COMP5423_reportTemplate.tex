\documentclass[12pt,twoside,letterpaper]{article}

\newcommand{\reporttitle}{Group X Project Report}
\newcommand{\reportauthors}{
Runyang YOU (24051234G), \\
Xin ZHANG (24054321G), \\
...
}

\newcommand{\reporttype}{Coursework}
\bibliographystyle{plain}

% load packages and define macros
\input{settings_do_not_modify/includes} % various packages needed for maths etc.
\input{notation} 


%%%%%%%%%%%%%%%%%%%%%%%%%%%%

\begin{document}
% front page
% Last modification: 2016-09-29 (Marc Deisenroth)
% Modification for UW: 2017-05-22 (jphickey)
\begin{titlepage}

\newcommand{\HRule}{\rule{\linewidth}{0.5mm}} % Defines a new command for the horizontal lines, change thickness here


%----------------------------------------------------------------------------------------
%	LOGO SECTION
%----------------------------------------------------------------------------------------



\begin{center} % Center remainder of the page

%----------------------------------------------------------------------------------------
%	HEADING SECTIONS
%----------------------------------------------------------------------------------------

\includegraphics[width = 10cm]{./figures/uw}\\[1.5cm] 
\textbf{\textsc{\Large COMP5423 - Natural Language Processing}}\\[1.0cm] 
\textsc{\Large Hong Kong Polytechnic University}\\[0.5cm] 
\textsc{\large Department of COMPUTING}\\[0.95cm] 

%----------------------------------------------------------------------------------------
%	TITLE SECTION
%----------------------------------------------------------------------------------------

\HRule \\[0.4cm]

{ \huge \bfseries \reporttitle}\\ % Title of your document
\HRule \\[1.5cm]
\end{center}
%----------------------------------------------------------------------------------------
%	AUTHOR SECTION
%----------------------------------------------------------------------------------------

%\begin{minipage}{0.4\hsize}
\begin{flushleft} \large
{Group Members:}\\
\reportauthors
\end{flushleft}
\vspace{4cm}
\makeatletter
Date: \@date

\vfill % Fill the rest of the page with whitespace



\makeatother


\end{titlepage}



%%%%%%%%%%%%%%%%%%%%%%%%%%%% Main document
\section*{Note:}
\emph{This document is intended to provide a sample structure for the reports in COMP5423 at the Hong Kong Polytechnic University. }

\section{Summary of the problem}
\emph{Describe the physical problem under investigation and the briefly summarize the governing equations. Example:}

We will turn our attention to the fascinating physical phenomenon of sonoluminescence. This occurs when a single bubble, usually of micrometer in size, within a liquid emits a short burst of light when imploding under an externally excited acoustic source. As the energy is concentrated into a point source, local temperatures in the collapsing bubble can reach up to 10,000 K for up to 50 pico seconds and visible light is emitted.

The origin of the light emission is an unsolved physical problem. For project 1, we will be studying the governing equations behind the bubble oscillation leading up to a the light burst.
\begin{figure}[h!]
\centering
\includegraphics[width=0.45\textwidth]{figures/goose.png} 
\caption{A goose.}
\label{goose}
\end{figure}


The radius of a bubble under a varying pressure field is defined by the Rayleigh-Plesset equation. This equation is derived using standard conservation law under a number of simplifying assumptions:
\begin{equation}
\rho_l \left(R\ddot{R} + \frac{3}{2}\dot{R}^2\right) = p_{gas} -P(t) -P_0 +\frac{R}{c}\frac{d p_{gas}}{dt} - 4\mu \frac{\dot{R}}{R}-\frac{2\sigma}{R}
\label{RP}
\end{equation}
In the above equation $R(t)$ [m] represents the bubble radius. The other terms are: $\rho_l$ [$kg\cdot m^3$]  the density of the liquid, $p_{gas}$ [Pa] the pressure of the gas inside the bubble, $P(t)$ [Pa] the imposed oscillatory pressure field, $P_0$ [Pa] the baseline pressure, $c$ [$m/s$] speed of sound in the liquid, $\mu$ [$Pa\cdot s$] molecular viscosity of the liquid and $\sigma$ [$kg\cdot s^{-2}$] the surface tension at the bubble-water interface.

For more contextual information on the bubble dynamic phenomena, please see the following sources: \cite{Hilgenfeldt1998}, \cite{Kreider2011}, and \cite{Lohse2003}.



\section{Questions}
\emph{This section answers the individual questions of the project description.  For each question, provide an answer and short analysis.}
\subsection*{Question 1: Forced bubble oscillation}
\subsubsection*{(a)} \emph{References to equations can be written out in latex \eqref{RP}. Similarly, figures  \ref{goose} and sections \ref{Q2} may also be referenced.}
\subsubsection*{(b)} \emph{Citations require the mybib.tex file to be extended with the desired references. Students can use JabRef (\url{http://www.jabref.org}) to construct the mybib.bib file. Students are invited to link their Mendelay, CiteULike and Zotero account directly to Overleaf. }

\subsection{Question 2: Bubble evolution \label{Q2}}


\vfill

% contributions page
\newpage
\section*{Contribution and Declaration}

\subsection*{Contribution}


The table below specifies the contribution roles and percentage allocation for each group member:

\begin{itemize}
    \item Runyang YOU (30\%): Responsible for ...
    \item Xin ZHANG (25\%): Contributed to...
    \item Student C (25\%): Worked on...
    \item Student D (20\%): Handled...
\end{itemize}

\subsection*{Declaration}



We, signed members of Group X, hereby make the following declarations:

\begin{itemize}


\item We confirm that the work presented in this project is the original result of our collective efforts, except where explicitly acknowledged through proper citation.

\item We guarantee the authenticity of all data, the accuracy of all analyses, and the integrity of the results presented herein. We affirm that the work has been conducted in accordance with accepted academic principles of the university.

\item We have collectively reviewed and critically evaluated all materials submitted for this project, acknowledge and fully agree with the contribution distribution stated above, and hereby give our full approval to this final version of submission.

\item We understand and acknowledge that any form of academic misconduct, including but not limited to plagiarism, fabrication, falsification, or other practices that violate academic integrity norms, will be subject to disciplinary actions in line with university policies.

\end{itemize}

By signing below, each member confirms their consent to the statements above:

\begin{center}
\begin{tabular}{cccc}
\includegraphics[width=0.20\textwidth]{figures/sign1.png} &
\includegraphics[width=0.20\textwidth]{figures/sign1.png} &
\includegraphics[width=0.20\textwidth]{figures/sign1.png} &
\includegraphics[width=0.20\textwidth]{figures/sign1.png} \\
Runyang YOU & Xin ZHANG & Student C & Student D \\
\end{tabular}
\end{center}


\vfill



\newpage
\bibliography{mybib}


\end{document}
%%% Local Variables: 
%%% mode: latex
%%% TeX-master: t
%%% End: 
